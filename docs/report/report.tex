\documentclass{vldb}

%%% Packages
\usepackage{graphicx}
\usepackage{balance}
\usepackage{tikz}
\usepackage{standalone}
\usepackage{subfigure}
\usepackage{colortbl}

%%% TikZ libraries
\usetikzlibrary{backgrounds, positioning, decorations.pathreplacing, calc, fit}
\usetikzlibrary{shapes}

%%% Macros
\newcommand{\tbr}{\textbf{[TO BE REVISED]}}
\newcommand{\tbd}{\textbf{[TO BE DONE]}}

\newcommand\diag[4]{%
  \multicolumn{1}{p{#2}|}{\hskip-\tabcolsep
  $\vcenter{\begin{tikzpicture}[baseline=0,anchor=south west,inner sep=#1]
  \path[use as bounding box] (0,0) rectangle (#2+2\tabcolsep,\baselineskip);
  \node[minimum width={#2+2\tabcolsep-\pgflinewidth},
        minimum  height=\baselineskip+\extrarowheight-\pgflinewidth] (box) {};
  \draw[line cap=round] (box.north west) -- (box.south east);
  \node[anchor=south west] at (box.south west) {#3};
  \node[anchor=north east] at (box.north east) {#4};
 \end{tikzpicture}}$\hskip-\tabcolsep}}

\begin{document}

% Title
\title{Building an Image Recognition System}
\subtitle{Project Report}

% Author(s)
\numberofauthors{1}

\author{
\alignauthor
	Daniel Kocher\\
  \affaddr{Department of Computer Sciences \\ University of Salzburg}\\
  \email{Daniel.Kocher@stud.sbg.ac.at}
}

\maketitle

\begin{abstract}
Automatic understanding of images (and videos) is a challenging problem in
computer vision. There exist different low- and high-level approaches to tackle
this problem.

In this paper, a project is presented which uses high-level a approach, i.e. an
attribute database. Out of this database, a Bag-of-Visual-Words (BoW)
representation is obtained and classifiers are trained for each attribute in the
database. The effectiveness of these classifiers is then evaluated.
\end{abstract}

\section{Introduction}
\label{sec:introduction}

Representing scenes can be done on different levels: there exist
low-level approaches, e.g. the \emph{spatial pyramid}~\cite{Lazebnik:2006}, and
higher-level approaches, like the SUN Attribute Database~\cite{Patterson:2012}.


\section{General approach}
\label{sec:general-approach}

\tbd

\subsection{High-Level Pipeline}
\label{subsec:high-level-pipeline}

\tbd

\subsection{Feature Extraction}
\label{subsec:feature-extraction}

\tbd

\subsection{Bag-of-Visual-Words}
\label{subsec:bovw}

\tbd

\subsection{Clustering}
\label{subsec:clustering}

\tbd

\section{Implementational Details}
\label{sec:implementational-details}

\tbd

\section{Experiments}
\label{sec:experiments}

\tbd

\section{Conclusion}
\label{sec:conclusion}

\tbd

%\clearpage
% balance columns on last page
\balance

\bibliographystyle{abbrv}
\bibliography{report}

% short hack to beautify last page
%\foreach \x in {1, ..., 6}{
%    \vphantom{1} \par
%}

\end{document}
